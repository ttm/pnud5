\documentclass[12pt]{article}
\usepackage[usenames,dvipsnames]{color}
\usepackage{listings}
\usepackage{graphicx}
\usepackage{fancyhdr}
\usepackage{framed}
\usepackage[T1]{fontenc}
\usepackage[toc,page]{appendix}
\usepackage[utf8]{inputenc}
\usepackage[brazil]{babel}
\usepackage{fancyvrb}
\usepackage[hmargin=2cm,vmargin=2cm]{geometry}
\usepackage{lastpage}
\usepackage{pdfpages}
\usepackage{makeidx}
\usepackage{hyperref}
\pagestyle{fancy}
\usepackage{enumitem}
% cabecalho e rodapé
\setlength{\headheight}{120pt}
\setlength{\textheight}{550pt}
\renewcommand{\headrulewidth}{0pt}
\lhead{\includegraphics[scale=0.03]{brasao.png}}
%\chead{\includegraphics[scale=0.5]{logo-brasil-sem-pobreza2.png}}
\rhead{\includegraphics[scale=0.5]{logo-pnud.png}}
\cfoot{\textbf{\ProjectCode\ - Inovando a democracia participativa}}
\rfoot{\thepage}

\hyphenation{par-ti-ci-pa-ção}
\bibliographystyle{ieeetr}

% definições sobre o autor e o produto
\newcommand{\MyName}{Renato Fabbri}
\newcommand{\MySurnameForename}{Fabbri, Renato}
\newcommand{\SupervisorName}{Ricardo Poppi}
\newcommand{\MyEmail}{renato.fabbri@gmail.com}
\newcommand{\ContractNumber}{2013/000566}
\newcommand{\ContractYear}{2014}
\newcommand{\ProjectCode}{Projeto BRA/12/018}
\newcommand{\NomeSecretaria}{Secretaria-Geral da Presidência da República}
%Q\newcommand{\SiglaSecretaria}{SG/PR}
\newcommand{\SiglaSecretaria}{Secretaria: SNAS }
\newcommand{\ProductNumber}{05}
\newcommand{\ProductTitle}{Proposta de regras de extração de conteúdos da API do portal e suas ferramentas para alimentação de eventual/hipotética base/nuvem de conhecimento de participação social}
\newcommand{\ProductSubtitle}{potencializando leituras focadas em incidência e participação social nas políticas públicas, com propostas de códigos}
\newcommand{\ProductDescription}{"Documento com proposta de regras de extração de conteúdos da API do portal e suas ferramentas para alimentação de eventual/hipotética base/nuvem de conhecimento de participação social potencializando leituras focadas em incidência e participação social nas políticas públicas, com propostas de códigos"}

\newcommand{\ProductValue}{R\$ 21,600 (vinte e um mil e seiscentos reais)}
\newcommand{\ObjetoContratacao}{
Aporte de conhecimentos e tecnologias para especificação de vocabulário e ferramentas assistidas que utilizam processamento de linguagem natural e análise de redes complexas para o conteúdo do portal da participação social.
}
\newcommand{\DataEntrega}{12 de Novembro de 2014}
\newcommand{\PalavrasChave}{reconhecimento de padrões, redes complexas, processamento de linguagem natural, web semântica, participação social}

% lista de abreviações
\makeindex
\begin{document}

\input{folhaderosto.tex}
\input{folhadeaprovacao.tex}
\input{folhadeidentificacao.tex}
\tableofcontents
\newpage


\begin{abstract}
Este documento descreve o quinto produto.

{\bf Palavras-chave:} \PalavrasChave.
\end{abstract}
\newpage

\section{Introdução}
\subsection{Contexto e importância da consultoria}
descrever o objetivo GERAL da consultoria e como este Produto específico está contextualizado dentro do objetivo final da contratação
\subsection{Contexto e importância do Produto}
\subsubsection{Objetivos}
\subsubsection{Resultados esperados}
\subsubsection{Caráter inovador}
destacar como este trabalho poderá contribuir suprir uma lacuna de conhecimento e/ou para desenvolver determinada a capacidade institucional da SG/PR.
\section{Desenvolvimento}\label{sec:dev}
espaço onde o consultor vai construir suas ideias. O consultor tem a liberdade para organizá-lo em tópicos, itens e sub-itens. 

demonstrar que o produto entregue corresponde ao que foi solicitado no termo de referência, por meio de:
4. Análise sobre os resultados esperados na etapa de planejamento do Produto e os resultados alcançados ao final do Produto.

\subsection{Etapas de desenvolvimento anteriores a este produto}
Descrição detalhada das etapas de desenvolvimento do Produto
\subsubsection{Sistematização ontológica da participação online}
Através de estudos e reuniões presenciais e online, a Ontologia de Participação Social (OPS) foi revisada~\cite{OPS} e a Ontologia do Participa.br (OPA) foi feita~\cite{OPA}.
\subsubsection{Triplificação dos dados do participa.br}
Feito um script para triplificar os dados do Participa.br, ou seja, para o enriquecimento semântico e escrita em RDF dos dados em Postgresql da instância Noosfero do Participa.br~\cite{triplifica}.
\subsubsection{Levantamento do endpoint SparQL}\label{sec:sfoo}
Para uso dos dados triplificados, pode-se recorrer a diversos métodos de leitura e disponibilização. Um método-chave é a disponibilização dos dados rdf (\emph{triple store}) em um \emph{endpoint sparql}. Para os fins de testes, pesquisa e usos leves, está disponibilizado um endpoint SparQL em servidores da USP~\cite{endpoint}.
\subsubsection{Análises iniciais, modelos}
Análises dos dados do participa.br foram abertas no IPython Notebook, com ênfase no texto produzido e nas redes formadas~\cite{repoProd3}.
\subsubsection{Sistema de recomendação de participante e recursos}
\subsection{Etapas de desenvolvimento deste produto}
\subsection{Justificativa, descrição detalhada e formas de aplicação do método}
\subsection{Justificativa, descrição detalhada e acesso das fontes}
\section{Resultados alcançados}
\subsection{Usos dos resultados}\label{sec:uso}
\section{Conclusão}
retomar as ideias trabalhadas ao longo do Produto e fazer uma análise sobre as mesmas.
\subsection{Comentários, sugestões, recomendações}
\subsection{Impacto do Produto para a elaboração, gestão e/ou avaliação de políticas públicas de participação social}
\subsection{Impacto no público-alvo das políticas públicas a que se refere}
\section{Agradecimentos}
O consultor Renato Fabbri agradece ao Joenio Costa pelo template em \LaTeX\ para os produtos. Agradece à Daniela Feitosa pela reunião para demanda de recomendação de perfis. Agradece aos supervisores do trabalho realizado em torno do participa.br: Ricardo Poppi e Ronald Costa. Agradece ao labMacambira.sf.net e todas as comunidades de software e cultura livre que compõe esta contribuição.
\newpage
\bibliography{bibliografia}
\newpage
%\listoffigures
\input{listadeabreviaturas.tex}
\newpage
\printindex
\newpage
%\input{listadeanexos.tex}
\appendix
\section{Ontologias de instâncias participativas online potencialmente relacionáveis ao participa.br}
\subsection{Ontologia do AA (Ontologiaa)}\label{ap:aa}
Como uma forma de integrar o Participa.br em uma nuvem de conhecimento participativo, foi levantada a Ontologiaa, exposta na Figura~\ref{fig:diaa}. O AA é uma técnica de compartilhamento de processos usada principalmente no labMacambira.sf.net. A simplicidade das implementações atuais, e a pertinência do registro e compartilhamento de processos, fizeram com que esta fosse o primeiro desenvolvimento efetivo deste último produto.

\begin{figure}[h!]
  \centering
    \includegraphics[width=\textwidth]{../figs/ontologiaa.png}
  \caption{Ontologia do AA, com suas classes, propriedades, literais, e classes e propriedades externas usadas para relacionar os dados do AA aos do participa e de toda nuvem LOD.}\label{fig:diaa}
\end{figure}

O tamanho reduzido da ontologia permitiu que vários testes fossem feitos. Em especial, com a ontologiaa foi reestabelecida a arquitetura de ontologia com uso de um namespace interno (no caso \url{http://purl.org/socialparticipation/aa/} e inferências para contemplar outros namespaces.

As inferências foram testadas com o jena/fuseki, com bons resultados. Tanto as inferências relacionadas às hiperonímias (superclasses e super propriedades, diretamente do rdfs) quanto inferências mais elaboradas (ligadas ao padrão OWL) foram satisfatórias. O revés é que qualquer query SparQL que demora milissegundos, mesmo que não envolva inferências para sua resposta, demora segundos quando há uma máquina de inferências ativa. A solução, portanto, parece ser ainda de realizar estas inferências offline e disponibilizar todas as triplas resultantes no endpoint.

Todos os desenvolvimentos desta ontologia e a triplificação de dados do AA em MySQL e MongoDB estão em: \url{https://github.com/ttm/aa01/tree/master/rdf}. Estes dados estão disponíveis no endpoint sparql (fuseki/jena) para uso conforme \url{script ipython}. Há interfaces úteis para explorar/expor os dados ligados ao AA. Em especial, estão derreferenciáveis, como na Figura~\ref{fig:aashout}.

\begin{figure}[h!]
  \centering
    \includegraphics[width=\textwidth]{../figs/aaShoutPubby.png}
  \caption{Mensagem (shout) do AA derreferenciado. Cada mensagem do AA recebe uma URI, assim como cada sessão e cada usuário. Estes três conceitos são instânciados com URIs dedicadas, e relacionadas via ainda outras URIs. Por fim URIs especificam relações entre instâncias destes conceitos e os dados.}\label{fig:aashout}
\end{figure}

A ontologia do AA está no webprotege da Stanford~\url{http://webprotege.stanford.edu/#Edit:projectId=5207dd13-8706-4836-bad9-6cba1c81de29}.

\subsection{Ontologia do Cidade Democrática (OCD)}
Outra instância participativa considerada prioritária pelo consultor para integração aos dados participativos linkados, e contemplada neste trabalho, foi o portal Cidade Democrática. Este portal possui grande complexidade e abundância de dados e conceitos. Assim, esta empreitada contrastou com a da Ontologiaa descrito no Apêndice~\ref{ap:aa}.

Com a grande complexidade das tabelas e dados, foi feita uma decupagem do banco de dados (disponibilizada em \url{https://github.com/ttm/ocd/blob/master/decupagemBD.txt}) e uma triplificação destes dados (script em: \url{https://github.com/ttm/ocd/blob/master/triplificaCD.py} e triplas resultantes em \url{https://github.com/ttm/ocd/blob/master/cdTriplestore.rdf.tar.gz}).

Embora os trabalhos de decupagem do banco e de triplificação dos dados sejam expressivos, o ponto alto desta empreitada foi a gênese de um método de levantamento de ontologia orientado aos dados. Este método é extremamente útil para qualquer portal que queira representar seus dados como triplas RDF e uma ontologia. O processo é o seguinte:

\begin{enumerate}
    \item Todos os dados de interesse são triplificados com namespace interno, conforme: \url{https://github.com/ttm/ocd/blob/master/triplificaCD.py}.
    \item Os dados triplificados são disponibilizados em um endpoint sparql para levantamento da ontologia com base nas triplas produzidas (endpoint em: \url{http://200.144.255.210:8082/cd/query}).
    \item Um script é construído, no qual os dados triplificados são usados para observação das estruturas ocorrentes, conforme \url{https://github.com/ttm/ocd/blob/master/OCD.py}. Principalmente:
\begin{itemize}
        \item São observadas todas as classes ocorrentes.
        \item São observadas todas as propriedades ocorrentes.
        \item As propriedades são especificadas como funcionais e inversamente funcionais (axiomas de propriedade), conforme os dados apresentarem tais relações.
        \item As classes recebem restrições universais e existenciais, conforme os dados apresentarem estas relações.
        \item São feitas imagens de cada propriedade, com os elementos imediatamente relacionados a eles, como na Figura~\ref{fig:ocdp}.
        \item São feitas imagens de cada classe, com os elementos imediatamente relacionados a eles, como na Figura~\ref{fig:ocdc}.
        \item São feitas imagens diferentes da estrutura global, para facilitar apreensão da ontologia, como a Figura~\ref{fig:ocdg}.
        \item Ontologia OWL é escrita, conforme disponibilizada em: \url{https://github.com/ttm/ocd/blob/master/OCD.owl} ou \url{https://github.com/ttm/ocd/blob/master/OCD.ttl}.
        \item Os conceitos são relacionados a conceitos mais gerais, de ontologias externas, para facilitar a integração dos dados do portal com o grafo gigante e global~\cite{LOD}. Este passo não foi dado na OCD por limitação de tempo mesmo. Há outras prioridades e a comunidade deste portal está ainda absorvendo as informações e tecnologias disponibilizadas com este produto.
\end{itemize}
\end{enumerate}

Esta ontologia está no Webprotege disponibilizado pela Stanford, no link: \url{http://webprotege.stanford.edu/#Edit:projectId=a6e2334c-5c32-4397-9c9a-c75c1cebb555} com todas as classes e propriedades comentáveis. Digno de nota: o detalhamento nas restrições de classes e nos axiomas de propriedade, embora usualmente não recomendados com tanta extensão para não forçar aplicação menos rígida, permitem mais inferencias. Além disso, este detalhamento melhorou bastante a navegação da estrutura, como pode-se observar no próprio webprotege.

\begin{figure}[h!]
  \centering
    \includegraphics[width=\textwidth]{../figs/topic.png}
  \caption{Figura da propriedade \texttt{ocd:topic}, fruto do método de especificação de ontologias orientado aos dados.}\label{fig:ocdp}
\end{figure}

\begin{figure}[h!]
  \centering
    \includegraphics[width=\textwidth]{../figs/Observatory.png}
  \caption{Figura da classe \texttt{ocd:Observatory}, fruto do método de especificação de ontologias orientado aos dados.}\label{fig:ocdc}
\end{figure}

\begin{figure}[h!]
  \centering
    \includegraphics[width=\textwidth]{../figs/OCD_2.png}
  \caption{Figura geral da ontologia \texttt{ocd}, fruto do método de especificação de ontologias orientado aos dados. Para facilitar visualização, visite as imagens diretamente em~\url{https://raw.githubusercontent.com/ttm/ocd/master/imgs/OCD_.png} e~\url{https://raw.githubusercontent.com/ttm/ocd/master/imgs/OCD_2.png}.}\label{fig:ocdg}
\end{figure}


\section{Revisão da OPa}\label{ap:opa}
O primeiro produto desta consultoria envolveu o levantamento de uma ontologia para o Participa, batizada de OPa. Isso ocorreu antes de serem triplificados os dados do participa.br e antes até mesmo do consultor ter acesso a estes dados. Isso foi bastante proveitoso, pois obrigou a equipe do participa a conceber uma ontologia genérica para portais participativos, centrada nos conceitos de participante, portal participativo e mecanismo participativo. Estes módulos estarão preservados e constam como legado intelectual, uma contribuição da equipe do participa.br na conceituação da participação social online.

Já depois de feitas estas ontologias do AA e do Cidade Democrática, e depois de ter feito toda a OBS e VBS, atingimos um paradigma apropriado para triplificação dos dados e organização conceitual. Em resumo:
\begin{itemize}
    \item uso de um namespace interno, como \texttt{http://purl.org/socialparticipation/opa}, para a triplificação dos dados, em todas as classes e propriedades. Isso facilita a navegacao e deixa os dados mais organizados, pois no caso extremo de compatibilidade com alguma classe externa, a classe interna assinala a fonte da instância. Um caso extremo desta pertinência é com o derreferenciamento, em que a url \url{http://purl.org/socialparticipation/opa/Participant} lista todos os participantes da OPA, e sua superclasse, \url{http://purl.org/socialparticipation/ops/Participant} apresenta todos os participantes, sejam da OPA, do AA ou do Cidade Democrática.
    \item Relacionamento das triplas com namespaces externos através da ontologia, com as propriedades \texttt{rdfs:subClassOf} e~\texttt{rdfs:subPropertyOf}. Esta implementação pode vir na medida em que a comunidade se apropriar do andamento, pois estas inferências tornam as consultas lentas pela utilização da máquina de inferência em tempo real ou aumentam a quantidade de triplas no caso das inferências offline. Ou seja, nos estágios iniciais, é mais leve e simples não utilizar namespaces externos.
    \item Liberação da ontologia com \emph{blueprints} em imagens. Acréscimo das restrições de classe, e axiomas de pripriedade na medida em que houver utilidade para não enrijecer a estrutura.
\end{itemize}

Neste contexto, além da ontologia disponível no primeiro produto desta consultoria, a OPa conta com as estruturas orientadas aos dados do portal, apresentadas nas imagens~\ref{fig:},~\ref{fig:},~\ref{fig:}.

\section{Revisão da Triplificação do Participa.br}
O script de triplificação disponibilizado no produto 2 foi adaptado para o paradigma explicitado no Apêndice~\ref{ap:ropa}. Além disso,
foram acrescentadas à triplificação algumas informações adicionais de usuários e das postagens em si. O script de triplificação, revisado, está em: \url{http://github.com/ttm/pnud5/scripts/triplificaParticipa22112014.py}.
\newpage
\section{Ontologia e Vocabulário da Biblioteca Social (OBS e VBS)}
Por ocasião do levantamento da Biblioteca (digital e semântica de participação) Social, por iniciativa da SNAS/SG-PR, e com esforços de diversos parceiros, o consultor iniciou uma sequência de entrevistas individuais. Um workshop foi feito na SGPR no dia 20/10/2014 para receber contribuições de parceiros diversos. Além das entrevistas individuais e do workshop, foram consideradas documentações de referência produzidas sobre e pelos mecanismos e instancias de participação social. A PNPS foi considerada separadamente, dada a informatividade do decreto. Todo o processo foi permeado de troca de mensagens entre o consultor e equipes do Particpa.br, UnB, IPEA, SNAS e MP.

Este apêndice expõe estas contribuições e as organizações ontológicas e de vocabulários que delas resultaram.

\subsection{Materiais enviados pela equipe para referência}
Foi considerado um material fruto de articulação da SNAS. O material consiste de documentos produzidos ou referentes às instâncias e mecanismos de participação social e um produto da consultora Carmen Romcy. Este material deu origem a um vocabulário SKOS sobre documentos de participação social, complementar aos obtidos nos apêndices seguintes. 
Veja a Tabela~\ref{tab:ovbs} para o script que escreve o vocabulário SKOS, para os arquivos RDF e para as listagens de termos.

O material em si está no link~\url{https://drive.google.com/file/d/0B7GnkNzm0kxvSTZWWDRyTFkzNm8/view?usp=sharing}.
 O decreto 8.243 é considerado no Apêndice~\ref{ap:pnps}.

\subsection{Entrevistas individuais e workshop}
Especialistas foram entrevistados individualmente para especificações ontológicas iniciais de instâncias de participação social e estão citados junto às instâncias sobre as quais ajudaram junto aos especialistas que contribuíram no workshop.

\subsubsection{Contextualização - Pedro Pontual, Carmen Romcy e Fernando Cruz}
As três entrevistas iniciais foram feitas com o diretor do DPS/SNAS Pedro Pontual, com a consultora Carmen Romcy (ambas no dia no dia 02 de outubro) e com o prof. Fernando Cruz (11 de outubro).

Na ocasião, foram feitas perguntas para contextualizar a OBS e VBS. Estas perguntas incluíam os usos que idealizavam os entrevistados para a biblioteca, possibilidades de alimentação, público alvo, informações disponibilizadas, técnicas e tecnologias.

Dada a extensão deste produto, este conteúdo não foi organizado nem revisado, e os entrevistados não foram contatados para a liberação das anotações destas entrevistas. O consultor manterá uma versão online em~\url{https://docs.google.com/document/d/1jJuzoQzgqPn_FUFVbleXeiKsXHD1Bd0D0O68nXXN7K8/edit?usp=sharing} para os fins deste documento e como registro do processo.

\subsubsection{Especificação das Conferências - Clovis Souza (entrevista), Clovis, Pedro, Fabiano (workshop)}
Materiais dos profs. Fernando Cruz e Carmem Romcy permitiram rascunhos iniciais da ontologia de conferências nacionais. Estes rascunhos foram usados como base para a entrevista feita com o Clovis Souza em dois dias: 30 de setembro e 03 de outubro. Foram revisados os rascunhos iniciais e especificadas relações dos documentos. Foram gerados dois conjuntos de conceitos relacionados ontologicamente e com vocabulário.

O primeiro conjunto é centrado na conferência em si, e os conceitos principais envolvidos. O segundo conjunto é centrado nos documentos e nos resultados das conferências. Veja a Tabela~\ref{tab:ovbs} para os scripts que escrevem as triplas RDF, os arquivos com ontologias OWL e vocabulários SKOS, e diagramas e listagens.

\subsubsection{Especificação dos Conselhos - Paula Pompeu (entrevista)}
No dia 16 de outubro, foi feita uma entrevista com Paula Pompeu para revisar os rascunhos sobre os conselhos nacionais.
Veja a Tabela~\ref{tab:ovbs} para os scripts que escrevem as triplas RDF, os arquivos com ontologias OWL e vocabulários SKOS, e diagramas e listagens.

\subsubsection{Especificação das Ouvidorias - Lígia M. A. Pereira (entrevista), Anjuli Osterne e Paulo Guimarães (workshop)}
No dia 21 de outubro de 2014, foi feita uma entrevista com Lígia Pereira para revistar a ontologia de ouvidorias federais.
Veja a Tabela~\ref{tab:ovbs} para os scripts que escrevem as triplas RDF, os arquivos com ontologias OWL e vocabulários SKOS, e diagramas e listagens.

\subsubsection{Especificação das Consultas públicas - Fernanda Lobato, Valéssio e Silvia (workshop)}
As contribuições dadas no workshop sobre as consultas públicas deram origem a uma ontologia e um vocabulário iniciais, conforme a Tabela~\ref{tab:ovbs}.

\subsubsection{Especificação das Mesas de diálogo - Roberto, Márcia (workshop)}
As contribuições dadas no workshop sobre as mesas de diálogo deram origem a uma ontologia e um vocabulário iniciais, conforme a Tabela~\ref{tab:ovbs}.


\subsection{Decreto 8.243 (PNPS)}\label{ap:pnps}
A observação cuidadosa do decreto 8.243 foi posterior à incorporação de todas as entrevistas individuais, contribuições de workshop e documentos enviados pela SNAS/SGPR. Ao contrário das ontologias anteriores, que eram usadas de base para o vocabulário, foi feito primeiro o vocabulário. Na sequência, a ontologia foi feita com uma nova leitura do decreto. Isso ocorreu após um estudo inicial, em que ontologias por demais detalhadas da PNPS não se mostraram apropriadas para os usos participativos a que se destinavam. Os estudos ficaram no papel, mas podem ser retomados, principalmente caso usos despontem para representações com detalhamento ontológico que contemplem o texto de forma mais minuciosa.

\subsection{Incorporação e registro das contribuições do workshop do dia \\20/Out/2014, sobre a biblioteca (semântica de participação) social}
Todas as contribuições entregues no workshop foram contempladas: ou integradas às OBS e VBS ou anotada com motivo para não estar integrada. As contribuições dos participantes foram escritas à caneta por eles sobre impressões em A4 e A3 dos diagramas das ontologias de conselhos e de conferências e sobre impressões em A4 das listagens dos vocabulários de conferências e conselhos. Fotos de cada uma das folhas com anotações dos participantes em caneta azul ou preta, e anotações do consultor em caneta vermelha, estão online em: \url{https://github.com/ttm/vocabulario-participacao/tree/master/figs/fotos}.

\subsection{Scripts adicionais}
Scripts para fins diversos, todos em \url{https://github.com/ttm/vocabulario-participacao/tree/master/scripts/}:
\begin{itemize}
    \item dspaceVocabLista.py - monta lista XML com todos os conceitos SKOS da VBS, no formato aceito pelo DSpace. Talvez seja útil algum detalhamento neste XML, o que precisa ser avaliado com os usuários/gestores.
    \item tematres.py - monta lista de termos da VBS para importação pelo tematres.
    \item tematresAutoridades.py - monta lista de autoridades da VBS para importação pelo tematres.
    \item triplificaConferencias.py - triplifica a tabela .xls com dados sobre conferências nacionais.
    \item triplificaConselhos.py -  triplifica a tabela .xls com dados sobre conselhos.
    \item vocabTexto.py - utiliza RDF da VBS para fazer os vocabulários em listas simples e enriquecidas de informações.
\end{itemize}

\subsection{Sistematização das instâncias e arquivos principais}\label{ap:sist}


\begin{table}[htpq!]
\vspace{-1cm}
\centering
\caption{Tabela de arquivos da OBS e VBS com o script que escreve os arquivos RDF (seja ontologia OWL ou vocabulário SKOS), o arquivo escrito em RDF/XML e em Turtle, diagramas para as ontologias, listagens para vocabulários, e uma instância do Webprotege com o recurso carregado e aberto para comentários e cópias. Veja os Apêndices~\ref{ap:ut} e~\ref{ap:sist} para mais informações.}
\begin{tabular}{| p{6cm} | c | c | c | c | c | }\hline
 {\bf descrição} & {\bf script} & {\bf RDF/XML} & {\bf Turtle} & {\bf diagramas ou listagens} & {\bf wp} \\\hline\hline
vocabulário SKOS do material de referência enviado pela SNAS  & \ref{i:1} & \ref{i:2} & \ref{i:3} & \ref{i:4},~\ref{i:4_1},~\ref{i:5} &~\ref{i:5wp} \\\hline\hline
ontologia OWL das conferências (Clovis) &~\ref{i:6}&~\ref{i:7}&~\ref{i:8}&~\ref{i:9},~\ref{i:10},~\ref{i:11},~\ref{i:11_1}&~\ref{i:11wp} \\
vocabulário SKOS das conferências (Clovis) &~\ref{i:12}&~\ref{i:13}&~\ref{i:14}&~\ref{i:16},~\ref{i:17},~\ref{i:18}&~\ref{i:18wp} \\\hline\hline

ontologia OWL  de documentos e resultados das conferências (Clovis) &~\ref{i:6a}&~\ref{i:7a}&~\ref{i:8a}&~\ref{i:9a},~\ref{i:10a},~\ref{i:11a},~\ref{i:11_1a}&~\ref{i:11awp} \\
vocabulário SKOS de documentos e resultados das conferências (Clovis) &~\ref{i:12a}&~\ref{i:13a}&~\ref{i:14a}&~\ref{i:16a},~\ref{i:17a},~\ref{i:18a}&~\ref{i:18awp} \\\hline\hline

ontologia OWL dos conselhos (Paula) &~\ref{i:19}&~\ref{i:20}&~\ref{i:21}&~\ref{i:22},~\ref{i:23},~\ref{i:24},~\ref{i:24_1}&~\ref{i:24wp} \\
vocabulário SKOS dos conselhos (Paula) &~\ref{i:25}&~\ref{i:26}&~\ref{i:27}&~\ref{i:28},~\ref{i:29},~\ref{i:30}&~\ref{i:30wp} \\\hline\hline

ontologia   OWL das  ouvidorias (Lígia) &~\ref{i:31}&~\ref{i:32}&~\ref{i:33}&~\ref{i:34},~\ref{i:35},~\ref{i:36},~\ref{i:37}&~\ref{i:37wp} \\
vocabulário SKOS das ouvidorias (Lígia) &~\ref{i:38}&~\ref{i:39}&~\ref{i:40}&~\ref{i:41},~\ref{i:42},~\ref{i:43}&~\ref{i:43wp} \\\hline\hline

ontologia   OWL  das consultas públicas (XX) &~\ref{i:44}&~\ref{i:45}&~\ref{i:46}&~\ref{i:47},~\ref{i:48},~\ref{i:49},~\ref{i:50}&~\ref{i:50wp} \\
vocabulário SKOS das consultas públicas (XX) &~\ref{i:51}&~\ref{i:52}&~\ref{i:53}&~\ref{i:54},~\ref{i:55},~\ref{i:56}&~\ref{i:56wp} \\\hline\hline

ontologia   OWL  das mesas de diálogo (XX) &~\ref{i:57}&~\ref{i:58}&~\ref{i:59}&~\ref{i:60},~\ref{i:61},~\ref{i:62},~\ref{i:63}&~\ref{i:63wp} \\
vocabulário SKOS das mesas de diálogo (XX) &~\ref{i:64}&~\ref{i:65}&~\ref{i:66}&~\ref{i:67},~\ref{i:68},~\ref{i:69}&~\ref{i:69wp} \\\hline\hline

ontologia   OWL  do Decreto 8.243 (PNPS) &~\ref{i:70}&~\ref{i:71}&~\ref{i:72}&~\ref{i:73},~\ref{i:75},{\bf ~\ref{i:76} - \ref{i:76ii} }&~\ref{i:76wp} \\
vocabulário SKOS do Decreto 8.243 (PNPS) &~\ref{i:77}&~\ref{i:78}&~\ref{i:79}&~\ref{i:80},~\ref{i:81},~\ref{i:82}&~\ref{i:82wp} \\\hline\hline

SKOS do vocabulário do IPEA de participação social &~\ref{i:83}&~\ref{i:84}&~\ref{i:85}&~\ref{i:86}&~\ref{i:86wp} \\\hline\hline
\end{tabular}\label{tab:ovbs}
\end{table}
\vspace{1cm}
Links da Tabela~\ref{tab:ovbs}:
{\scriptsize
\begin{enumerate}
    \item \url{https://raw.githubusercontent.com/ttm/vocabulario-participacao/master/scripts/vbsDocumentacaoVBS.py}\label{i:1}
    \item \url{https://raw.githubusercontent.com/ttm/vocabulario-participacao/master/rdf/vbsDocumentacaoVBS.rdf}\label{i:2}
    \item \url{https://raw.githubusercontent.com/ttm/vocabulario-participacao/master/rdf/vbsDocumentacaoVBS.ttl}\label{i:3}
    \item \url{https://raw.githubusercontent.com/ttm/vocabulario-participacao/master/txt/vbsDocumentacaoVBS.txt} \label{i:4}
    \item \url{https://raw.githubusercontent.com/ttm/vocabulario-participacao/master/txt/vbsDocumentacaoVBSPodada.txt}\label{i:4_1}
    \item \url{https://raw.githubusercontent.com/ttm/vocabulario-participacao/master/txt/vbsDocumentacaoVBSPalavras.txt}\label{i:5}
    \item \url{http://webprotege.stanford.edu/#Edit:projectId=cfe287a2-71a7-4f45-9387-5a4ff097647b}\label{i:5wp}

    \item \url{https://raw.githubusercontent.com/ttm/vocabulario-participacao/master/scripts/obsConferencias.py}\label{i:6}
    \item \url{https://raw.githubusercontent.com/ttm/vocabulario-participacao/master/rdf/obsConferencia.owl}\label{i:7}
    \item \url{https://raw.githubusercontent.com/ttm/vocabulario-participacao/master/rdf/obsConferencia.ttl}\label{i:8}
    \item \url{https://raw.githubusercontent.com/ttm/vocabulario-participacao/master/figs/obsConferencia.png}\label{i:9}
    \item \url{https://raw.githubusercontent.com/ttm/vocabulario-participacao/master/figs/obsConferencia.png}\label{i:10}
    \item \url{https://raw.githubusercontent.com/ttm/vocabulario-participacao/master/figs/obsConferencia.png}\label{i:11}
    \item \url{https://raw.githubusercontent.com/ttm/vocabulario-participacao/master/dot/obsConferencia.dot}\label{i:11_1}
    \item \url{http://webprotege.stanford.edu/#Edit:projectId=b9c725ac-e673-443f-97ef-dab340e4a94f}\label{i:11wp}

    \item \url{https://raw.githubusercontent.com/ttm/vocabulario-participacao/master/scripts/vbsConferencias.py}\label{i:12}
    \item \url{https://raw.githubusercontent.com/ttm/vocabulario-participacao/master/rdf/vbsConferencia.rdf}\label{i:13}
    \item \url{https://raw.githubusercontent.com/ttm/vocabulario-participacao/master/rdf/vbsConferencia.ttl}\label{i:14}
    \item \url{https://raw.githubusercontent.com/ttm/vocabulario-participacao/master/txt/vbsConferencia.txt}\label{i:16}
    \item \url{https://raw.githubusercontent.com/ttm/vocabulario-participacao/master/txt/vbsConferenciaPalavras.txt}\label{i:17}
    \item \url{https://raw.githubusercontent.com/ttm/vocabulario-participacao/master/txt/vbsConferenciaPodada.txt}\label{i:18}
    \item \url{http://webprotege.stanford.edu/#Edit:projectId=dd679187-69bf-40b9-a811-26644c75caf2}\label{i:18wp}

 \item \url{https://raw.githubusercontent.com/ttm/vocabulario-participacao/master/scripts/obsConferenciasDocsRes.py}    \label{i:6a}
     \item \url{https://raw.githubusercontent.com/ttm/vocabulario-participacao/master/rdf/obsConferenciaDocsRes.owl}    \label{i:7a}
     \item \url{https://raw.githubusercontent.com/ttm/vocabulario-participacao/master/rdf/obsConferenciaDocsRes.ttl}    \label{i:8a}
    \item \url{https://raw.githubusercontent.com/ttm/vocabulario-participacao/master/figs/obsConferenciaDocsRes.png}    \label{i:9a}
    \item \url{https://raw.githubusercontent.com/ttm/vocabulario-participacao/master/figs/obsConferenciaDocsRes.png}   \label{i:10a}
    \item \url{https://raw.githubusercontent.com/ttm/vocabulario-participacao/master/figs/obsConferenciaDocsRes.png}   \label{i:11a}
     \item \url{https://raw.githubusercontent.com/ttm/vocabulario-participacao/master/dot/obsConferenciaDocsRes.dot} \label{i:11_1a}
     \item \url{http://webprotege.stanford.edu/#Edit:projectId=4f78f256-4f43-4318-b3ee-4395e50a38fd} \label{i:11awp}

\item \url{https://raw.githubusercontent.com/ttm/vocabulario-participacao/master/scripts/vbsConferenciasDocsRes.py}         \label{i:12a}
     \item \url{https://raw.githubusercontent.com/ttm/vocabulario-participacao/master/rdf/vbsConferenciaDocsRes.rdf}        \label{i:13a}
     \item \url{https://raw.githubusercontent.com/ttm/vocabulario-participacao/master/rdf/vbsConferenciaDocsRes.ttl}        \label{i:14a}
     \item \url{https://raw.githubusercontent.com/ttm/vocabulario-participacao/master/txt/vbsConferenciaDocsRes.txt}        \label{i:16a}
     \item \url{https://raw.githubusercontent.com/ttm/vocabulario-participacao/master/txt/vbsConferenciaDocsResPalavras.txt}\label{i:17a}
     \item \url{https://raw.githubusercontent.com/ttm/vocabulario-participacao/master/txt/vbsConferenciaDocsResPodada.txt}  \label{i:18a}
     \item \url{http://webprotege.stanford.edu/#Edit:projectId=523e053a-6d9a-4baf-9ac9-b3c7f4b3f5a3}  \label{i:18awp}

    \item \url{https://raw.githubusercontent.com/ttm/vocabulario-participacao/master/scripts/obsConselhos.py}\label{i:19}
    \item  \url{https://raw.githubusercontent.com/ttm/vocabulario-participacao/master/rdf/obsConselho.owl}\label{i:20}
    \item  \url{https://raw.githubusercontent.com/ttm/vocabulario-participacao/master/rdf/obsConselho.ttl}\label{i:21}
    \item \url{https://raw.githubusercontent.com/ttm/vocabulario-participacao/master/figs/obsConselho.png}\label{i:22}
    \item \url{https://raw.githubusercontent.com/ttm/vocabulario-participacao/master/figs/obsConselho.png}\label{i:23}
    \item \url{https://raw.githubusercontent.com/ttm/vocabulario-participacao/master/figs/obsConselho.png}\label{i:24}
    \item \url{https://raw.githubusercontent.com/ttm/vocabulario-participacao/master/dot/obsConselho.dot}\label{i:24_1}
    \item \url{http://webprotege.stanford.edu/#Edit:projectId=b0e87a16-d33f-4ed4-86bd-e7c288f34215}\label{i:24wp}

    \item \url{https://raw.githubusercontent.com/ttm/vocabulario-participacao/master/scripts/vbsConselhos.py}\label{i:25}
    \item \url{https://raw.githubusercontent.com/ttm/vocabulario-participacao/master/rdf/vbsConselho.rdf}\label{i:26}
    \item \url{https://raw.githubusercontent.com/ttm/vocabulario-participacao/master/rdf/vbsConselho.ttl}\label{i:27}
    \item \url{https://raw.githubusercontent.com/ttm/vocabulario-participacao/master/txt/vbsConselho.txt}\label{i:28}
    \item \url{https://raw.githubusercontent.com/ttm/vocabulario-participacao/master/txt/vbsConselhoPalavras.txt}\label{i:29}
    \item \url{https://raw.githubusercontent.com/ttm/vocabulario-participacao/master/txt/vbsConselhoPodada.txt}\label{i:30}
    \item \url{http://webprotege.stanford.edu/#Edit:projectId=f921bc20-3650-4e64-a7ad-a83b1723b5a9}\label{i:30wp}

    \item \url{https://raw.githubusercontent.com/ttm/vocabulario-participacao/master/scripts/obsOuvidorias.py}\label{i:31}
    \item  \url{https://raw.githubusercontent.com/ttm/vocabulario-participacao/master/rdf/obsOuvidoria.owl}\label{i:32}
    \item  \url{https://raw.githubusercontent.com/ttm/vocabulario-participacao/master/rdf/obsOuvidoria.ttl}\label{i:33}
    \item \url{https://raw.githubusercontent.com/ttm/vocabulario-participacao/master/figs/obsOuvidoria.png}\label{i:34}
    \item \url{https://raw.githubusercontent.com/ttm/vocabulario-participacao/master/figs/obsOuvidoria.png}\label{i:35}
    \item \url{https://raw.githubusercontent.com/ttm/vocabulario-participacao/master/figs/obsOuvidoria.png}\label{i:36}
    \item  \url{https://raw.githubusercontent.com/ttm/vocabulario-participacao/master/dot/obsOuvidoria.dot}\label{i:37}
    \item  \url{http://webprotege.stanford.edu/#Edit:projectId=f8968196-5f12-4b52-acae-9bcd89e75449}\label{i:37wp}

    \item \url{https://raw.githubusercontent.com/ttm/vocabulario-participacao/master/scripts/vbsOuvidorias.py}\label{i:38}
    \item \url{https://raw.githubusercontent.com/ttm/vocabulario-participacao/master/rdf/vbsOuvidoria.rdf}\label{i:39}
    \item \url{https://raw.githubusercontent.com/ttm/vocabulario-participacao/master/rdf/vbsOuvidoria.ttl}\label{i:40}
    \item \url{https://raw.githubusercontent.com/ttm/vocabulario-participacao/master/txt/vbsOuvidoria.txt}\label{i:41}
    \item \url{https://raw.githubusercontent.com/ttm/vocabulario-participacao/master/txt/vbsOuvidoriaPalavras.txt}\label{i:42}
    \item \url{https://raw.githubusercontent.com/ttm/vocabulario-participacao/master/txt/vbsOuvidoriaPodada.txt}\label{i:43}
    \item \url{http://webprotege.stanford.edu/#Edit:projectId=673e6bd0-f814-4d9c-abb5-fc61b377e95e}\label{i:43wp}

 \item \url{https://raw.githubusercontent.com/ttm/vocabulario-participacao/master/scripts/obsConsultas.py}\label{i:44}
    \item  \url{https://raw.githubusercontent.com/ttm/vocabulario-participacao/master/rdf/obsConsulta.owl}\label{i:45}
    \item  \url{https://raw.githubusercontent.com/ttm/vocabulario-participacao/master/rdf/obsConsulta.ttl}\label{i:46}
    \item \url{https://raw.githubusercontent.com/ttm/vocabulario-participacao/master/figs/obsConsulta.png}\label{i:47}
    \item \url{https://raw.githubusercontent.com/ttm/vocabulario-participacao/master/figs/obsConsulta.png}\label{i:48}
    \item \url{https://raw.githubusercontent.com/ttm/vocabulario-participacao/master/figs/obsConsulta.png}\label{i:49}
    \item  \url{https://raw.githubusercontent.com/ttm/vocabulario-participacao/master/dot/obsConsulta.dot}\label{i:50}
    \item  \url{http://webprotege.stanford.edu/#Edit:projectId=d1ebd08c-a88e-48f6-a92c-54664573ba87}\label{i:50wp}

\item \url{https://raw.githubusercontent.com/ttm/vocabulario-participacao/master/scripts/vbsConsultas.py}        \label{i:51}
    \item \url{https://raw.githubusercontent.com/ttm/vocabulario-participacao/master/rdf/vbsConsulta.rdf}        \label{i:52}
    \item \url{https://raw.githubusercontent.com/ttm/vocabulario-participacao/master/rdf/vbsConsulta.ttl}        \label{i:53}
    \item \url{https://raw.githubusercontent.com/ttm/vocabulario-participacao/master/txt/vbsConsulta.txt}        \label{i:54}
    \item \url{https://raw.githubusercontent.com/ttm/vocabulario-participacao/master/txt/vbsConsultaPalavras.txt}\label{i:55}
    \item \url{https://raw.githubusercontent.com/ttm/vocabulario-participacao/master/txt/vbsConsultaPodada.txt}  \label{i:56}
    \item \url{http://webprotege.stanford.edu/#Edit:projectId=c03fba0c-75ff-4382-87c2-c86660cd7228}  \label{i:56wp}

 \item \url{https://raw.githubusercontent.com/ttm/vocabulario-participacao/master/scripts/obsMesasDeDialogo.py}\label{i:57}
    \item  \url{https://raw.githubusercontent.com/ttm/vocabulario-participacao/master/rdf/obsMesaDeDialogo.owl}\label{i:58}
    \item  \url{https://raw.githubusercontent.com/ttm/vocabulario-participacao/master/rdf/obsMesaDeDialogo.ttl}\label{i:59}
    \item \url{https://raw.githubusercontent.com/ttm/vocabulario-participacao/master/figs/obsMesaDeDialogo.png}\label{i:60}
    \item \url{https://raw.githubusercontent.com/ttm/vocabulario-participacao/master/figs/obsMesaDeDialogo.png}\label{i:61}
    \item \url{https://raw.githubusercontent.com/ttm/vocabulario-participacao/master/figs/obsMesaDeDialogo.png}\label{i:62}
    \item  \url{https://raw.githubusercontent.com/ttm/vocabulario-participacao/master/dot/obsMesaDeDialogo.dot}\label{i:63}
    \item  \url{ihttp://webprotege.stanford.edu/#Edit:projectId=b76fbe91-6688-4973-92a1-a77bc09df8eb}\label{i:63wp}

\item \url{https://raw.githubusercontent.com/ttm/vocabulario-participacao/master/scripts/vbsMesasDeDialogo.py}        \label{i:64}
    \item \url{https://raw.githubusercontent.com/ttm/vocabulario-participacao/master/rdf/vbsMesaDeDialogo.rdf}        \label{i:65}
    \item \url{https://raw.githubusercontent.com/ttm/vocabulario-participacao/master/rdf/vbsMesaDeDialogo.ttl}        \label{i:66}
    \item \url{https://raw.githubusercontent.com/ttm/vocabulario-participacao/master/txt/vbsMesaDeDialogo.txt}        \label{i:67}
    \item \url{https://raw.githubusercontent.com/ttm/vocabulario-participacao/master/txt/vbsMesaDeDialogoPalavras.txt}\label{i:68}
    \item \url{https://raw.githubusercontent.com/ttm/vocabulario-participacao/master/txt/vbsMesaDeDialogoPodada.txt}  \label{i:69}
    \item \url{http://webprotege.stanford.edu/#Edit:projectId=a0f33da0-3c37-480e-855c-6a9e544535e2}  \label{i:69wp}

 \item \url{https://raw.githubusercontent.com/ttm/vocabulario-participacao/master/scripts/obsPNPS.py} \label{i:70}
    \item  \url{https://raw.githubusercontent.com/ttm/vocabulario-participacao/master/rdf/obsPNPS.owl}\label{i:71}
    \item  \url{https://raw.githubusercontent.com/ttm/vocabulario-participacao/master/rdf/obsPNPS.ttl}\label{i:72}
    \item \url{https://raw.githubusercontent.com/ttm/vocabulario-participacao/master/figs/obsPNPS.png}\label{i:73}
    \item \url{https://raw.githubusercontent.com/ttm/vocabulario-participacao/master/figs/obsPNPS3.png}\label{i:75}

    \item  \url{https://raw.githubusercontent.com/ttm/vocabulario-participacao/master/dot/obsPNPS.dot}\label{i:76}
    \item  \url{https://raw.githubusercontent.com/ttm/vocabulario-participacao/master/dot/obsPNPS_ambientev.png}\label{i:76b}
    \item  \url{https://raw.githubusercontent.com/ttm/vocabulario-participacao/master/dot/obsPNPS_ambientev2.png}\label{i:76c}
    \item  \url{https://raw.githubusercontent.com/ttm/vocabulario-participacao/master/dot/obsPNPS_ambientev3.png}\label{i:76d}
    \item  \url{https://raw.githubusercontent.com/ttm/vocabulario-participacao/master/dot/obsPNPS_audiencia.png}\label{i:76e}
    \item  \url{https://raw.githubusercontent.com/ttm/vocabulario-participacao/master/dot/obsPNPS_audiencia2.png}\label{i:76f}
    \item  \url{https://raw.githubusercontent.com/ttm/vocabulario-participacao/master/dot/obsPNPS_audiencia3.png}\label{i:76g}
    \item  \url{https://raw.githubusercontent.com/ttm/vocabulario-participacao/master/dot/obsPNPS_comissao.png}\label{i:76h}
    \item  \url{https://raw.githubusercontent.com/ttm/vocabulario-participacao/master/dot/obsPNPS_comissao2.png}\label{i:76i}
    \item  \url{https://raw.githubusercontent.com/ttm/vocabulario-participacao/master/dot/obsPNPS_comissao3.png}\label{i:76j}
    \item  \url{https://raw.githubusercontent.com/ttm/vocabulario-participacao/master/dot/obsPNPS_conferencia.png}\label{i:76k}
    \item  \url{https://raw.githubusercontent.com/ttm/vocabulario-participacao/master/dot/obsPNPS_conferencia2.png}\label{i:76l}
    \item  \url{https://raw.githubusercontent.com/ttm/vocabulario-participacao/master/dot/obsPNPS_conferencia3.png}\label{i:76m}
    \item  \url{https://raw.githubusercontent.com/ttm/vocabulario-participacao/master/dot/obsPNPS_conselho.png}\label{i:76n}
    \item  \url{https://raw.githubusercontent.com/ttm/vocabulario-participacao/master/dot/obsPNPS_conselho2.png}\label{i:76o}
    \item  \url{https://raw.githubusercontent.com/ttm/vocabulario-participacao/master/dot/obsPNPS_conselho3.png}\label{i:76p}
    \item  \url{https://raw.githubusercontent.com/ttm/vocabulario-participacao/master/dot/obsPNPS_consulta.png}\label{i:76q}
    \item  \url{https://raw.githubusercontent.com/ttm/vocabulario-participacao/master/dot/obsPNPS_consulta2.png}\label{i:76r}
    \item  \url{https://raw.githubusercontent.com/ttm/vocabulario-participacao/master/dot/obsPNPS_consulta3.png}\label{i:76s}
    \item  \url{https://raw.githubusercontent.com/ttm/vocabulario-participacao/master/dot/obsPNPS_forumInterconselhos.png}\label{i:76t}
    \item  \url{https://raw.githubusercontent.com/ttm/vocabulario-participacao/master/dot/obsPNPS_forumInterconselhos2.png}\label{i:76u}
    \item  \url{https://raw.githubusercontent.com/ttm/vocabulario-participacao/master/dot/obsPNPS_forumInterconselhos3.png}\label{i:76v}
    \item  \url{https://raw.githubusercontent.com/ttm/vocabulario-participacao/master/dot/obsPNPS_mesa.png}\label{i:76x}
    \item  \url{https://raw.githubusercontent.com/ttm/vocabulario-participacao/master/dot/obsPNPS_mesa2.png}\label{i:76w}
    \item  \url{https://raw.githubusercontent.com/ttm/vocabulario-participacao/master/dot/obsPNPS_mesa3.png}\label{i:76z}
    \item  \url{https://raw.githubusercontent.com/ttm/vocabulario-participacao/master/dot/obsPNPS_mesam.png}\label{i:76aa}
    \item  \url{https://raw.githubusercontent.com/ttm/vocabulario-participacao/master/dot/obsPNPS_mesam2.png}\label{i:76bb}
    \item  \url{https://raw.githubusercontent.com/ttm/vocabulario-participacao/master/dot/obsPNPS_mesam3.png}\label{i:76cc}
    \item  \url{https://raw.githubusercontent.com/ttm/vocabulario-participacao/master/dot/obsPNPS_ouvidoria.png}\label{i:76dd}
    \item  \url{https://raw.githubusercontent.com/ttm/vocabulario-participacao/master/dot/obsPNPS_ouvidoria2.png}\label{i:76ee}
    \item  \url{https://raw.githubusercontent.com/ttm/vocabulario-participacao/master/dot/obsPNPS_ouvidoria3.png}\label{i:76ff}
    \item  \url{https://raw.githubusercontent.com/ttm/vocabulario-participacao/master/dot/obsPNPS_preliminar.png}\label{i:76gg}
    \item  \url{https://raw.githubusercontent.com/ttm/vocabulario-participacao/master/dot/obsPNPS_preliminar2.png}\label{i:76hh}
    \item  \url{https://raw.githubusercontent.com/ttm/vocabulario-participacao/master/dot/obsPNPS_preliminar3.png}\label{i:76ii}

    \item  \url{http://webprotege.stanford.edu/#Edit:projectId=1f8948f8-fa58-490e-9e81-79457c79a4a3}\label{i:76wp}

\item \url{https://raw.githubusercontent.com/ttm/vocabulario-participacao/master/scripts/vbsPNPS.py}         \label{i:77}
    \item \url{https://raw.githubusercontent.com/ttm/vocabulario-participacao/master/rdf/vbsPNPS.rdf}        \label{i:78}
    \item \url{https://raw.githubusercontent.com/ttm/vocabulario-participacao/master/rdf/vbsPNPS.ttl}        \label{i:79}
    \item \url{https://raw.githubusercontent.com/ttm/vocabulario-participacao/master/txt/vbsPNPS.txt}        \label{i:80}
    \item \url{https://raw.githubusercontent.com/ttm/vocabulario-participacao/master/txt/vbsPNPSPalavras.txt}\label{i:81}
    \item \url{https://raw.githubusercontent.com/ttm/vocabulario-participacao/master/txt/vbsPNPSPodada.txt}  \label{i:82}
    \item \url{http://webprotege.stanford.edu/#Edit:projectId=e5c6b505-1c78-4cae-8424-8e628d070da6}  \label{i:82wp}

\item \url{https://raw.githubusercontent.com/ttm/vocabulario-participacao/master/scripts/vbsIPEA.py}         \label{i:83}
    \item \url{https://raw.githubusercontent.com/ttm/vocabulario-participacao/master/rdf/vbsIPEA.rdf}        \label{i:84}
    \item \url{https://raw.githubusercontent.com/ttm/vocabulario-participacao/master/rdf/vbsIPEA.ttl}        \label{i:85}
    \item \url{https://raw.githubusercontent.com/ttm/vocabulario-participacao/master/txt/vbsIPEAPalavras.txt}\label{i:86}
    \item \url{http://webprotege.stanford.edu/#Edit:projectId=11cf0d1a-b13e-43bc-b651-9d95471a692a}\label{i:86wp}
\end{enumerate}
}

\section{Utilização dos dados linkados}\label{ap:ut}

As ontologias mais importantes para este trabalho precisam ser observáveis e anotáveis com facilidade por não especialistas.

Uma primeira opção é aproveitar os dados/triplas disponíveis no endpoint sparql já aberto com uma instância Fuseki/Jena. Este endpoint pode ser acessado via linguagens de scripting para prototipação rápida, como JavaScript, Ruby ou Python, fornecendo interfaces gráficas e web para navegação e análise, com acesso pleno aos dados triplificados via buscas semânticas. Estas possibilidades estão desenvolvidas nos produtos 2, 3 e 4 desta mesma consultoria~\cite{repoProd2, repoProd3, repoProd4}. A OBS e VBS estão disponíveis no endpoint \url{http://200.144.255.210:8082/participacao/query}. Dados do AA, e suas relações ontológicas, estão disponíveis em \url{http://200.144.255.210:8082/aa/query}. Dados do Cidade Democrática, e suas relações ontológicas, estão disponíveis em \url{http://200.144.255.210:8082/cidadedemocratica/query}. Dados do Participa.br, e suas relações ontológidas, estão disponíveis em \url{http://200.144.255.210:8082/participabr/query}. Uma interface geral está em \url{http://200.144.255.210:8082/}. No momento, todos os dados e ontologias estão disponiveis em  \url{http://200.144.255.210:8082/participacao/query} junto à OBS e VBS, para facilitar experimentos.

\subsection{Pubby}
Com desenvolvimento recente no github \url{https://github.com/cygri/pubby}, é talvez o navegador de dados mais conhecido. Parece ser projeto do dig (grupo do Berners-Lee no MIT).
Os testes mostraram que as consultas sparql demoravam demais para o montante de dados triplificados, portanto foram importados os rdfs no pubby diretamente. O arquivo de configuração do pubby está em \url{https://github.com/ttm/vocabulario-participacao/blob/master/auxiliar/config.ttl}.

A url permanente \url{http://purl.org/socialparticipation/} é redirecionada para a instância do pubby, em: \url{http://200.144.255.210:8081/tpubby/page/}. Desta forma, \emph{todos} os conceitos da VBS e classes da OBS podem ser derreferenciados, como na Figura~\ref{fig:derre}. Todos os conceitos, classes e instâncias da OPA, OPS e AA estão também derreferenciados, como na Figura~\ref{fig:aashout}, através do redirecionamento do prefixo \url{http://purl.prg/socialparticipation/} para o endereço \url{http://200.144.255.210:8081/tpubby/page/}, onde opera o pubby.

\begin{figure}[h!]
  \centering
    \includegraphics[width=\textwidth]{../figs/derre.png}
  \caption{Derreferenciamento do conceito de Instância de participação através da URI \url{http://purl.org/socialparticipation/obs/ParticipationInstance}.}\label{fig:derre}
\end{figure}

O pubby está operante em uma máquina de pesquisa da nuvem da USP, cuja administração é feita pelo consultor com a permissão de técnicos e docentes do IFSC/USP. Embora útil, a porta não padrão para navegação HTTP impede que a maioria dos gestores públicos federais acessem estes links e derreferenciem as URIs para obter informações sobre o vocábulo, classe ontológica, propriedade que relaciona um objeto a um dado ou outro objeto, instâncias de classes, etc. Colaboradores e funcionários podem estudar e testar o pubby e outras intefaces para dados linkados para que sejam disponibilizados servicos melhores e mais diversificados.

\subsection{Webprotege}

Este software permite que as ontologias e vocabulários estejam online e comentáveis, além da criação e edição de ontologias na própria interface.

Nas tentativas do consultor de levantar uma instância deste software, foram encontrados alguns roteiros incompletos na wiki e conflitos do java. Dados os resultados instáveis e especificidades técnicas, o consultor subiu os arquivos no webprotege instalado e disponibilizado pela equipe da Stanford que desenvolve o webprotege. 
No presente momento, apenas o consultor pode editar, mas todos podem comentar ou fazer cópias dos vocabulários e ontologias. Veja a Tabela~\ref{tab:ovbs} para os links.





%\section{Cartas ao poder público federal}
%\subsection{Religião e cultura livre}
%\subsection{Autoritarismo/preconceito e rede social}
%\subsection{Pesquisa e desenvolvimento para participação}
%\subsection{Subsídios para pulverizar a participação e qualificá-la}

%\section{Técnicas de difusão de informação}
%\subsection{Reprodução do experimento apocalíptico de 2012}




\end{document}
