\documentclass[12pt]{article}
\usepackage[usenames,dvipsnames]{color}
\usepackage{listings}
\usepackage{graphicx}
\usepackage{fancyhdr}
\usepackage{framed}
\usepackage[T1]{fontenc}
\usepackage[toc,page]{appendix}
\usepackage[utf8]{inputenc}
\usepackage[brazil]{babel}
\usepackage{fancyvrb}
\usepackage[hmargin=2cm,vmargin=2cm]{geometry}
\usepackage{lastpage}
\usepackage{pdfpages}
\usepackage{makeidx}
\usepackage{hyperref}
\pagestyle{fancy}
\usepackage{enumitem}
% cabecalho e rodapé
\setlength{\headheight}{120pt}
\setlength{\textheight}{550pt}
\renewcommand{\headrulewidth}{0pt}
\lhead{\includegraphics[scale=0.03]{brasao.png}}
%\chead{\includegraphics[scale=0.5]{logo-brasil-sem-pobreza2.png}}
\rhead{\includegraphics[scale=0.5]{logo-pnud.png}}
\cfoot{\textbf{\ProjectCode\ - Inovando a democracia participativa}}
\rfoot{\thepage}

\hyphenation{par-ti-ci-pa-ção}
\bibliographystyle{ieeetr}

% definições sobre o autor e o produto
\newcommand{\MyName}{Renato Fabbri}
\newcommand{\MySurnameForename}{Fabbri, Renato}
\newcommand{\SupervisorName}{Ricardo Poppi}
\newcommand{\MyEmail}{renato.fabbri@gmail.com}
\newcommand{\ContractNumber}{2013/000566}
\newcommand{\ContractYear}{2014}
\newcommand{\ProjectCode}{Projeto BRA/12/018}
\newcommand{\NomeSecretaria}{Secretaria-Geral da Presidência da República}
%Q\newcommand{\SiglaSecretaria}{SG/PR}
\newcommand{\SiglaSecretaria}{Secretaria: SNAS }
\newcommand{\ProductNumber}{05}
\newcommand{\ProductTitle}{Proposta de regras de extração de conteúdos da API do portal e suas ferramentas para alimentação de eventual/hipotética base/nuvem de conhecimento de participação social}
\newcommand{\ProductSubtitle}{potencializando leituras focadas em incidência e participação social nas políticas públicas, com propostas de códigos}
\newcommand{\ProductDescription}{"Documento com proposta de regras de extração de conteúdos da API do portal e suas ferramentas para alimentação de eventual/hipotética base/nuvem de conhecimento de participação social potencializando leituras focadas em incidência e participação social nas políticas públicas, com propostas de códigos"}

\newcommand{\ProductValue}{R\$ 21,600 (vinte e um mil e seiscentos reais)}
\newcommand{\ObjetoContratacao}{
Aporte de conhecimentos e tecnologias para especificação de vocabulário e ferramentas assistidas que utilizam processamento de linguagem natural e análise de redes complexas para o conteúdo do portal da participação social.
}
\newcommand{\DataEntrega}{12 de Novembro de 2014}
\newcommand{\PalavrasChave}{reconhecimento de padrões, redes complexas, processamento de linguagem natural, web semântica, participação social}

% lista de abreviações
\makeindex
\begin{document}

\input{folhaderosto.tex}
\input{folhadeaprovacao.tex}
\input{folhadeidentificacao.tex}
\tableofcontents
\newpage


\begin{abstract}
Este documento descreve o quinto produto.

{\bf Palavras-chave:} \PalavrasChave.
\end{abstract}
\newpage

\section{Introdução}
\subsection{Contexto e importância da consultoria}
descrever o objetivo GERAL da consultoria e como este Produto específico está contextualizado dentro do objetivo final da contratação
\subsection{Contexto e importância do Produto}
\subsubsection{Objetivos}
\subsubsection{Resultados esperados}
\subsubsection{Caráter inovador}
destacar como este trabalho poderá contribuir suprir uma lacuna de conhecimento e/ou para desenvolver determinada a capacidade institucional da SG/PR.
\section{Desenvolvimento}\label{sec:dev}
espaço onde o consultor vai construir suas ideias. O consultor tem a liberdade para organizá-lo em tópicos, itens e sub-itens. 

demonstrar que o produto entregue corresponde ao que foi solicitado no termo de referência, por meio de:
4. Análise sobre os resultados esperados na etapa de planejamento do Produto e os resultados alcançados ao final do Produto.

\subsection{Etapas de desenvolvimento anteriores a este produto}
Descrição detalhada das etapas de desenvolvimento do Produto
\subsubsection{Sistematização ontológica da participação online}
Através de estudos e reuniões presenciais e online, a Ontologia de Participação Social (OPS) foi revisada~\cite{OPS} e a Ontologia do Participa.br (OPA) foi feita~\cite{OPA}.
\subsubsection{Triplificação dos dados do participa.br}
Feito um script para triplificar os dados do Participa.br, ou seja, para o enriquecimento semântico e escrita em RDF dos dados em Postgresql da instância Noosfero do Participa.br~\cite{triplifica}.
\subsubsection{Levantamento do endpoint SparQL}\label{sec:sfoo}
Para uso dos dados triplificados, pode-se recorrer a diversos métodos de leitura e disponibilização. Um método-chave é a disponibilização dos dados rdf (\emph{triple store}) em um \emph{endpoint sparql}. Para os fins de testes, pesquisa e usos leves, está disponibilizado um endpoint SparQL em servidores da USP~\cite{endpoint}.
\subsubsection{Análises iniciais, modelos}
Análises dos dados do participa.br foram abertas no IPython Notebook, com ênfase no texto produzido e nas redes formadas~\cite{repoProd3}.
\subsubsection{Sistema de recomendação de participante e recursos}
\subsection{Etapas de desenvolvimento deste produto}
\subsection{Justificativa, descrição detalhada e formas de aplicação do método}
\subsection{Justificativa, descrição detalhada e acesso das fontes}
\section{Resultados alcançados}
\subsection{Usos dos resultados}\label{sec:uso}
\section{Conclusão}
retomar as ideias trabalhadas ao longo do Produto e fazer uma análise sobre as mesmas.
\subsection{Comentários, sugestões, recomendações}
\subsection{Impacto do Produto para a elaboração, gestão e/ou avaliação de políticas públicas de participação social}
\subsection{Impacto no público-alvo das políticas públicas a que se refere}
\section{Agradecimentos}
O consultor Renato Fabbri agradece ao Joenio Costa pelo template em \LaTeX\ para os produtos. Agradece à Daniela Feitosa pela reunião para demanda de recomendação de perfis. Agradece aos supervisores do trabalho realizado em torno do participa.br: Ricardo Poppi e Ronald Costa. Agradece ao labMacambira.sf.net e todas as comunidades de software e cultura livre que compõe esta contribuição.
\newpage
\bibliography{bibliografia}
\newpage
%\listoffigures
\input{listadeabreviaturas.tex}
\newpage
\printindex
\newpage
%\input{listadeanexos.tex}
\appendix
\section{Ontologias de instâncias participativas online potencialmente relacionáveis ao participa.br}
\subsection{Ontologia do AA (Ontologiaa)}\label{ap:aa}
Como uma forma de integrar o Participa.br em uma nuvem de conhecimento participativo, foi levantada a Ontologiaa, exposta na Figura~\ref{fig:diaa}. O AA é uma técnica de compartilhamento de processos usada principalmente no labMacambira.sf.net. A simplicidade das implementações atuais, e a pertinência do registro e compartilhamento de processos, fizeram com que esta fosse o primeiro desenvolvimento efetivo deste último produto.

\begin{figure}[h!]
  \centering
    \includegraphics[width=\textwidth]{../figs/ontologiaa.png}
  \caption{Ontologia do AA, com suas classes, propriedades, literais, e classes e propriedades externas usadas para relacionar os dados do AA aos do participa e de toda nuvem LOD.}\label{fig:diaa}
\end{figure}

O tamanho reduzido da ontologia permitiu que vários testes fossem feitos. Em especial, com a ontologiaa foi reestabelecida a arquitetura de ontologia com uso de um namespace interno (no caso \url{http://purl.org/socialparticipation/aa/} e inferências para contemplar outros namespaces.

As inferências foram testadas com o jena/fuseki, com bons resultados. Tanto as inferências relacionadas às hiperonímias (superclasses e super propriedades, diretamente do rdfs) quanto inferências mais elaboradas (ligadas ao padrão OWL) foram satisfatórias. O revés é que qualquer query SparQL que demora milissegundos, mesmo que não envolva inferências para sua resposta, demora segundos quando há uma máquina de inferências ativa. A solução, portanto, parece ser ainda de realizar estas inferências offline e disponibilizar todas as triplas resultantes no endpoint.

Todos os desenvolvimentos desta ontologia e a triplificação de dados do AA em MySQL e MongoDB estão em: \url{https://github.com/ttm/aa01/tree/master/rdf}. Estes dados estão disponíveis no endpoint sparql (fuseki/jena) para uso conforme \url{script ipython}. Há interfaces úteis para explorar/expor os dados ligados ao AA. Em especial, estão derreferenciáveis, como na Figura~\ref{fig:aashout}.

\begin{figure}[h!]
  \centering
    \includegraphics[width=\textwidth]{../figs/aaShoutPubby.png}
  \caption{Mensagem (shout) do AA derreferenciado. Cada mensagem do AA recebe uma URI, assim como cada sessão e cada usuário. Estes três conceitos são instânciados com URIs dedicadas, e relacionadas via ainda outras URIs. Por fim URIs especificam relações entre instâncias destes conceitos e os dados.}\label{fig:aashout}
\end{figure}

\subsection{Ontologia do Cidade Democrática (OCD)}
Outra instância participativa considerada prioritária pelo consultor para integração aos dados participativos linkados, e contemplada neste trabalho, foi o portal Cidade Democrática. Este portal possui grande complexidade e abundância de dados e conceitos. Assim, esta empreitada contrastou com a da Ontologiaa descrito no Apêndice~\ref{ap:aa}.

Com a grande complexidade das tabelas e dados, foi feita uma decupagem do banco de dados (disponibilizada em \url{https://github.com/ttm/ocd/blob/master/decupagemBD.txt}) e uma triplificação destes dados (script em: \url{https://github.com/ttm/ocd/blob/master/triplificaCD.py} e triplas resultantes em \url{https://github.com/ttm/ocd/blob/master/cdTriplestore.rdf.tar.gz}).

Embora os trabalhos de decupagem do banco e de triplificação dos dados sejam expressivos, o ponto alto desta empreitada foi a gênese de um método de levantamento de ontologia orientado aos dados. Este método é extremamente útil para qualquer portal que queira representar seus dados como triplas RDF e uma ontologia. O processo é o seguinte:

\begin{enumerate}
    \item Todos os dados de interesse são triplificados com namespace interno, conforme: \url{https://github.com/ttm/ocd/blob/master/triplificaCD.py}.
    \item Os dados triplificados são disponibilizados em um endpoint sparql para levantamento da ontologia com base nas triplas produzidas (endpoint em: \url{http://200.144.255.210:8082/cd/query}).
    \item Um script é construído, no qual os dados triplificados são usados para observação das estruturas ocorrentes, conforme \url{https://github.com/ttm/ocd/blob/master/OCD.py}. Principalmente:
\begin{itemize}
        \item São observadas todas as classes ocorrentes.
        \item São observadas todas as propriedades ocorrentes.
        \item São feitas imagens de cada classe e de cada propriedade, com os elementos imediatamente relacionados a eles, como nas Figuras~\ref{} e~\ref{}.
\end{itemize}
\end{enumerate}
\section{Revisão da OPA}
\section{Revisão da Triplificação do Participa.br}
\section{Ontologia e Vocabulário da Biblioteca Social (OBS e VBS)}
\subsection{PNPS}
\subsection{Entrevistas}
\subsection{Relatório de implementação das contribuições do Workshop dia 20/Out/2014, sobre a biblioteca (semântica de participação) social}
\begin{figure}[h!]
  \centering
    \includegraphics[width=\textwidth,angle=270]{fotos/OuvidoriaLigia.jpg}
  \caption{Diagrama de Ouvidorias desenhado com o acompanhamento de especialista (Lígia).}
\end{figure}
\subsection{Materiais enviados pela equipe para referência}
\section{Utilização dos dados linkados}
As ontologias mais importantes para este trabalho precisam ser observáveis e anotáveis com facilidade por não especialistas.

Uma primeira opção é aproveitar os dados/tripas disponíveis no endpoint sparql já aberto com uma instância Fuseki/Jena. Este endpoint pode ser acessado via linguagens de scripting para prototipação rápida, como JavaScript ou Python, fornecendo interfaces gráficas e web para navegação e análise. Estas possibilidades estão desenvolvidas nos produtos 2 e 3 desta mesma consultoria~\cite{prod2, prod3}.

\subsection{Pubby}

Com desenvolvimento recente no github \url{https://github.com/cygri/pubby}, é talvez o navegador de dados mais conhecido. Parece ser projeto do dig (grupo do Berners-Lee no MIT).
Os testes mostraram que as consultas sparql demoravam demais para o montante de dados triplificados, portanto foram importados os rdfs. Para facilitar, o arquivo de configuração do pubby está em \url{https://github.com/ttm/vocabulario-participacao/blob/master/auxiliar/config.ttl}.

A url permanente \url{http://purl.org/socialparticipation/} é redirecionada para a instância do pubby, em: \url{http://200.144.255.210:8081/tpubby/page/}. Desta forma, \emph{todos} os conceitos da VBS e classes da OBS podem ser derreferenciados, como na Figura~\ref{fig:derre}.

\begin{figure}[h!]
  \centering
    \includegraphics[width=\textwidth]{../figs/derre.png}
  \caption{Derreferenciamento do conceito de Instância de participação através da URI \url{http://purl.org/socialparticipation/obs/ParticipationInstance}.}\label{fig:derre}
\end{figure}

\subsection{Endpoint SparQL}

Os dados da OBS, VBS, Participabr, AA e OCD estão disponíveis via endpoint SparQL, conforme o uso explicitado nos scripts do INotebook \url{http://200.144.255.210:8003/}.

\subsection{Webprotege}

Permite que as ontologias e vocabulários estejam online e comentáveis.

%\section{Cartas ao poder público federal}
%\subsection{Religião e cultura livre}
%\subsection{Autoritarismo/preconceito e rede social}
%\subsection{Pesquisa e desenvolvimento para participação}
%\subsection{Subsídios para pulverizar a participação e qualificá-la}

%\section{Técnicas de difusão de informação}
%\subsection{Reprodução do experimento apocalíptico de 2012}




\end{document}
